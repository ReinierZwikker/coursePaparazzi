\chapter{Installing and Running Paparazzi}
This chapter will guide you through the installation procedure of Ubuntu and the Paparazzi autopilot.
In the following chapters you will perform your first flight using Paparazzi, make the drone autonomous and implement your own algorithms.

\section{Installing Ubuntu}
For this course you are required to install Ubuntu.
If you are already running Ubuntu 20.04 or 22.04, skip to the next section.
Otherwise, follow the steps here to install Ubuntu in a dual-boot setup with your existing Windows installation.
This allows you to keep using your existing Windows installation next to Ubuntu.
While it is possible to install Ubuntu as a virtual machine, this frequently causes problems with networking, USB and GPU support. Therefore we do not support virtual machines during the course.

The dual-boot installation of Ubuntu requires you to format parts of your drive, therefore there is a small risk of losing important files.
Make sure you have backed up your files before starting the installation!

Follow this tutorial to install Ubuntu next to an existing Windows installation: \url{https://itsfoss.com/install-ubuntu-1404-dual-boot-mode-windows-8-81-uefi/}. 

We recommend you install Ubuntu 22.04 \url{https://www.ubuntu.com/download/desktop}. The rest of the course manuals assume that you are running Ubuntu 22.04. 

\textbf{NOTE} Ubuntu 24.04 is not supported at this time!

Some useful tips and common issues:

\begin{itemize}
  \item It may be useful to look up which function key (or combination of keys) your laptop manufacturer uses to enter the BIOS during start-up
  
  \item Sometimes the system will not let you shrink a partition's volume, even though free space is available. This is caused by immovable files placed towards the end of the volume (see \url{https://en.wikipedia.org/wiki/File\_system\_fragmentation}). Windows or third-party defragmentation utilities can help fix this problem. \textbf{NOTE} that these tools could leave your system unable to boot, so a system backup is strongly advised!
  
  \item If the Secure Boot option is grayed out in the BIOS, set the supervisor password in the BIOS
  
  \item After installing Ubuntu, if your computer boots directly to Windows, check the UEFI settings in your BIOS (usually accessible by pressing F10 or another FN key right after powering on the laptop) and change the boot order from "Windows Boot Manager" to "ubuntu" or "\textit{name of your hard disk}"
  
  \item Some BIOS let you add a delay at start-up to give you time to select which OS you would like to boot from. This can usually be changed in the BIOS Setup.
  
  \item \textbf If you cannot manage to boot from the USB with the Ubuntu installation image it may be necessary to switch from UEFI Native to UEFI Legacy (With CSM) in the BIOS setup. The USB should then become visible in the Boot Device Options at startup (usually F9 during power-on)
\end{itemize}


\section{Installing Paparazzi}\label{sec:install}
Once Ubuntu is running, the next step is to install Paparazzi.
We have prepared a Github branch \href{https://github.com/tudelft/paparazzi/tree/\coursebranch}{here} that contains the Paparazzi autopilot and example files for the course. The rest of this manual will show you how to install Paparazzi.

Most steps require you to use Ubuntu's Command Line Interface (CLI), also known as the \emph{terminal}. You can open a terminal window by pressing Ctrl+Alt+T. Any commands that you need to run are listed as follows:

\begin{lstlisting}[style=Bash]
	command 1
	command 2
\end{lstlisting}

You need to type or copy these commands \emph{line-by-line} to the terminal, pressing enter after each line. In other words: do not copy \emph{all} commands at once, this can sometimes produce unexpected results. Note that pasting into the terminal uses Ctrl+\emph{Shift}+V, not Ctrl+V.

Use the following steps to download the course branch and install Paparazzi:

\begin{enumerate}
	
	\item During the course we will use \href{https://git-scm.com/}{git} to keep track of your code changes and to collaborate with your teammates. First, go to GitHub (\url{https://github.com/}) and set up an account, if you do not have one already. 
	
	\item Then, open a terminal (Ctrl+Alt+T) and install git. 
	
	\begin{lstlisting}[style=Bash]
		sudo apt-get update
		sudo apt-get install git
	\end{lstlisting}

	\item In the terminal, register your git credentials using
	\begin{lstlisting}[style=Bash]
		git config --global user.name !\hl{your\_name}!
		git config --global user.email !\hl{your\_email@host.com}!
	\end{lstlisting}
	where you should replace \texttt{your\_name} and \texttt{your\_email@host.com} with the username and email you have registered at \url{https://github.com/}.

	\item First, the TU Delft repository of Paparazzi needs to be cloned. In the folder you want to download the code, typically your home folder, do:

	\begin{lstlisting}[style=Bash]
		git clone https://github.com/tudelft/paparazzi.git
		cd ./paparazzi
	\end{lstlisting}

	\item To collaborate with your team, one team member should fork the repository by going to the tudelft paparazzi branch on \href{https://github.com/tudelft/paparazzi/}{https://github.com/tudelft/paparazzi/} and searching for the FORK button. This will create a copy for your team under \href{https://github.com/your_name/paparazzi/}{https://github.com/\hl{your\_name}/paparazzi/}, where \texttt{your\_name} is again your GitHub username. The person that forked the repository should give the other team members access via the \url{https://github.com/} interface.

	Then the new copy should be added on all team computers with:
	\begin{lstlisting}[style=Bash]
		git remote add team1 git@github.com/!\hl{your\_name}!/paparazzi.git
	\end{lstlisting}

	Before you can write to this private copy, you need to be able to authenticate by installing an SSH key: \href{https://docs.github.com/en/authentication/connecting-to-github-with-ssh/adding-a-new-ssh-key-to-your-github-account}{https://docs.github.com/en/authentication/connecting-to-github-with-ssh/adding-a-new-ssh-key-to-your-github-account}

	\item Checkout (Go to) the \coursebranch{} branch using:
	\begin{lstlisting}[style=Bash]
		git checkout !\coursebranch!
	\end{lstlisting}



	\item Now install Paparazzi by running the instal script:
	\begin{lstlisting}[style=Bash]
		./install.sh
	\end{lstlisting}

	Then install all required items one by one. Check after each item if there were no errors during the installation. If you run into trouble, also check \url{http://wiki.paparazziuav.org/wiki/Installation}

	If this fails, you can manually install paparazzi by following these steps:

	\begin{enumerate}
		\item Initialize, sync and update Paparazzi's submodules. Submodules are other git projects included in this project. Use:
		\begin{lstlisting}[style=Bash]
			git submodule init
			git submodule sync
			git submodule update
		\end{lstlisting}

		\item Add the dependency repository:
		\begin{lstlisting}[style=Bash]
			sudo add-apt-repository -y ppa:paparazzi-uav/ppa
			sudo apt-get update
		\end{lstlisting}

		\item Install all dependencies:
		\begin{lstlisting}[style=Bash]
			sudo apt-get install -y python3-lxml python3-numpy
			sudo apt-get install -y paparazzi-dev
			sudo apt-get install -y python-is-python3 gcc-arm-none-eabi gdb-multiarch
			sudo apt-get install -y paparazzi-jsbsim
			sudo apt-get install -y pprzgcs
			sudo apt-get install -y python3-pyqt5.qtwebkit
			sudo apt-get install -y dfu-util
			sudo cp conf/system/udev/rules/*.rules /etc/udev/rules.d/ && sudo udevadm control --reload-rules
		\end{lstlisting}
	
		\item If you are using Ubuntu 20.04, install Gazebo 9:
		\begin{lstlisting}[style=Bash]
			sudo apt-get install -y gazebo9 libgazebo9-dev
		\end{lstlisting}

		Otherwise, if you are using Ubuntu 22.04, install Gazebo 11:
		\begin{lstlisting}[style=Bash]
			sudo apt-get install -y gazebo libgazebo-dev
		\end{lstlisting}
		% Include this?
		Gazebo 11 is not available for Ubuntu 24.04, which is why this version of Ubuntu is not supported for this course.

		\item Next to Paparazzi you will need some additional tools. Install ffmpeg, vlc, cmake, jstest-gtk and java using:
		\begin{lstlisting}[style=Bash]
			sudo apt install ffmpeg vlc cmake jstest-gtk default-jre
		\end{lstlisting}

		\item Build Paparazzi by using:
		\begin{lstlisting}[style=Bash]
			make clean
			make -j1
		\end{lstlisting}

		\item Run the first-time setup script:
		\begin{lstlisting}[style=Bash]
			./start.py
		\end{lstlisting}
	\end{enumerate}
	
	\item If the installation was succesful, there should be a screen to select a configuration and control panel. Select the Conf and Control Panel files that were prepared for the course:
	\begin{enumerate}
	\item select as Conf: \fbox{userconf/tudelft/course\_conf.xml}\\
	\item select as Controlpanel: \fbox{userconf/tudelft/course\_control\_panel.xml}.\\
	\end{enumerate}
	Click `Set active' and close the dialog by clicking `Exit'.
	

	% \item Install the Gazebo simulator. Gazebo version 9 is recommended for Ubuntu 18.04 and 20.04. For Ubuntu 22.04 the version is Gazeobo 11. 
	% Install Gazebo 9 using:
	% \begin{lstlisting}[style=Bash]
	% 	sudo apt install gazebo9 libgazebo9-dev
	% \end{lstlisting}
	% Install Gazebo 11 (unbuntu 22.04) using:
	% \begin{lstlisting}[style=Bash]
	% 	sudo apt install gazebo libgazebo-dev
	% \end{lstlisting}
	
	\item Install a good editor like Visual Studio Code.
		
	\begin{enumerate}
		\item{Install via Ubuntu Software}
		\item{Do ``add folder to workspace'' and open the paparazzi folder.}
		\item{Ctrl-p lets you open a file}
		\item{Ctrl-Shift-F lets you search in files}
	\end{enumerate}
	
	\item Optional: Build OpenCV for the Bebop if you plan to use it. Note that OpenCV is heavy and thereby slow compared to self
	optimized C-code. That is why the example does not use OpenCV.

	\begin{enumerate}
		\item Navigate to paparazzi/sw/ext/opencv\_bebop:
		\begin{lstlisting}[style=Bash]
			cd ~/paparazzi/sw/ext/opencv_bebop
		\end{lstlisting}
		\item Install the required OpenCV libraries
		\begin{lstlisting}[style=Bash]
			sudo apt install libjpeg-turbo8-dev libpng-dev libtiff-dev zlib1g-dev libdc1394-22-dev
		\end{lstlisting}
		\item Make openCV using:
		\begin{lstlisting}[style=Bash]
			make
		\end{lstlisting}
	\end{enumerate}

\end{enumerate}


\section{Running your first simulation}

To test your Paparazzi installation, you will perform a short test-flight in simulation.

\begin{enumerate}

\item Navigate to the Paparazzi folder and launch the Paparazzi Center using

\begin{lstlisting}[style=Bash]
	cd ~/paparazzi
	./paparazzi
\end{lstlisting}

\item In the top left drop-down box, make sure that the `bebop\_orange\_avoid' aircraft is selected.

\item In the top middle under `Build', select `nps' in the `Target' dropdown.

\item Click `Clean' \inlineicon{icons/clean.png}, then `Build' \inlineicon{icons/build.png} to compile the example code.

\item When the compilation is finished, go to the `Operation' tab.

\item Under `Control Panel', make sure that `userconf/tudelft/course\_control\_panel.xml' is selected. 

\item Under `Session', select `Simulation - Gazebo' and click `Start Session' \inlineicon{icons/start.png}. You should now see the Ground Control Station and the Gazebo simulator.

\item Click `Stop All' \inlineicon{icons/stop.png} in the Paparazzi Center to end the simulation.

\end{enumerate}

You should now have a working installation of paparazzi!