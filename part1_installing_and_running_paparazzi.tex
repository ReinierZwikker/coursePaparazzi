\documentclass{article}

\usepackage[margin=3cm]{geometry}

\usepackage{fancyhdr}
\pagestyle{fancy}
\usepackage{graphicx}
\usepackage{color}
\usepackage{xcolor}
\usepackage{hyperref}
\usepackage{cleveref}
\usepackage{tcolorbox}
%% Define no indent with line skip with new paragraph
\usepackage{parskip}

\definecolor{code}{rgb}{.92,.92,.99}
\definecolor{codered}{rgb}{.92,.62,.62}
\definecolor{darkgreen}{rgb}{.2,.62,.18}
\usepackage{listings}
\lstset{xrightmargin=10pt,xleftmargin=10pt,language=Java,captionpos=b,tabsize=3,frame=none,keywordstyle=\color{blue},commentstyle=\color{darkgreen},stringstyle=\color{red},showstringspaces=false,basicstyle=\footnotesize\ttfamily,emph={label},backgroundcolor=\color{code}}
%%numbers=left,numberstyle=\tiny,numbersep=5pt,breaklines=true,\

\usepackage{soul}
\lstdefinestyle{Bash}
{language=bash,
  escapechar=!,
  backgroundcolor=\color{white},
  columns=fullflexible,
  breaklines=true,         
  breakatwhitespace=false,
  inputencoding=utf8x,
  keywordstyle=\color{black},
  basicstyle=\small\ttfamily,
%  morekeywords={peter@kbpet},
%  alsoletter={:~$},
%  morekeywords=[2]{peter@kbpet:},
%  keywordstyle=[2]{\color{red}},
%  literate={\$}{{\textcolor{red}{\$}}}1 
%         {:}{{\textcolor{red}{:}}}1
%         {~}{{\textcolor{red}{\textasciitilde}}}1,
  literate={~}{\raisebox{-0.5ex}{\textasciitilde}}1,
}

\newenvironment{code}
{\begin{minipage}[l]{\textwidth}}
{\end{minipage}}

\newtcolorbox{mybox}{colback=blue!5!white,colframe=blue!75!black}

%\newcommand{\ccpmaketitle}[4][Roland Meertens, Christophe de Wagter, Guido de Croon, Tom van Dijk] {
%	\author{#1}
%	\title{\bf Crash Course Paparazzi 2020\\#2\\{\large #4#3}}
%	\date{February 2020}
%	\setlength{\parindent}{0em}
%	\maketitle
%}
\def\coursebranch{mavlabCourse\the\year}

% Set headers and footers
\fancyhead[LO,LE]{Crash Course Paparazzi \the\year}
\renewcommand{\chaptermark}[1]{\markboth{\chaptername\ \thechapter.\ #1}{}}
\fancyfoot[LO,LE]{{\small Crash Course Paparazzi \the\year}}
\fancyfoot[CO,CE]{\thepage}


\begin{document}
\ccpmaketitle[Tom van Dijk, Kirk Scheper]{Installing and running Paparazzi}{}{Part 1}

In this part you will install Ubuntu and the Paparazzi autopilot.
In the following parts you will perform your first flight using Paparazzi, make the drone autonomous and implement your own algorithms.


\section{Installing Ubuntu}
For this course you are required to install Ubuntu.
If you are already running Ubuntu 18.04, skip to the next section.
Otherwise, follow the steps here to install Ubuntu in a dual-boot setup with your existing Windows installation.
This allows you to keep using your existing Windows installation next to Ubuntu.
While it is possible to install Ubuntu as a virtual machine, this frequently causes problems with networking, USB and GPU support. Therefore we do not support virtual machines during the course.

The dual-boot installation of Ubuntu requires you to format parts of your drive, therefore there is a small risk of losing important files.
Make sure you have backed up your files before starting the installation.

Follow this tutorial to install Ubuntu next to an existing Windows installation: \url{https://itsfoss.com/install-ubuntu-1404-dual-boot-mode-windows-8-81-uefi/}. We recommend you install Ubuntu 18.04 Bionic Beaver \url{https://www.ubuntu.com/download/desktop}. The rest of the course manuals assume that you are running Ubuntu 18.04.



\section{Installing Paparazzi}
Once Ubuntu is running, the next step is to install Paparazzi.
We have prepared a github branch \href{https://github.com/tudelft/paparazzi/tree/mavlabCourse2019}{here} that contains the Paparazzi autopilot and example files for the course. The rest of this manual will show you how to install Paparazzi.

Most steps require you to use Ubuntu's Command Line Interface (CLI), also known as the \emph{terminal}. You can open a terminal window by pressing Ctrl+Alt+T. Any commands that you need to run are listed as follows:
\begin{lstlisting}[style=Bash]
command 1
command 2
\end{lstlisting}
You need to type or copy these commands \emph{line-by-line} to the terminal, pressing enter after each line. In other words: do not copy \emph{all} commands at once, this can sometimes produce unexpected results. Note that pasting into the terminal uses Ctrl+\emph{Shift}+V, not Ctrl+V.

Use the following steps to download the course branch and install Paparazzi:
\begin{enumerate}
\item{Install Paparazzi UAV by opening a terminal (Ctrl+Alt+T) and running the one-liner found here: \url{http://wiki.paparazziuav.org/wiki/Installation}. Copy and paste (Ctrl+Shift+V) the entire one-liner into the terminal to prevent typing errors, then press enter to run it. After running the one-liner, you should see the Paparazzi center. Close this and return to the terminal.}
\item{During the course we will use \href{https://git-scm.com/}{git} to keep track of your code changes and to collaborate with your teammates. First, go to \url{https://github.com/} and set up an account if you do not have one already. Then, in the terminal, register your git credentials using
\begin{lstlisting}[style=Bash]
git config --global user.name !\hl{your\_name}!
git config --global user.email !\hl{your\_email@host.com}!
\end{lstlisting}
where you should replace \texttt{your\_name} and \texttt{your\_email@host.com} with the username and email you have registered at github.com.}
\item{Navigate to the paparazzi folder and add the tudelft remote.
\begin{lstlisting}[style=Bash]
cd ~/paparazzi
git remote add tudelft https://github.com/tudelft/paparazzi
git fetch tudelft mavlabCourse2019
\end{lstlisting}
}
\item{Checkout the mavlabCourse2019 branch using:
\begin{lstlisting}[style=Bash]
git checkout mavlabCourse2019
\end{lstlisting}
}
\item{Initialize, sync and update Paparazzi's submodules using:
\begin{lstlisting}[style=Bash]
git submodule init
git submodule sync
git submodule update
\end{lstlisting}
}
\item{Build Paparazzi by using:
\begin{lstlisting}[style=Bash]
make clean
make
\end{lstlisting}
\item{Select the conf and control panel files that were prepared for the course by running:
\begin{lstlisting}[style=Bash]
python start.py
\end{lstlisting}
and select as Conf: \fbox{userconf/tudelft/course2019\_conf.xml}\\
and as Controlpanel: \fbox{userconf/tudelft/course2019\_control\_panel.xml}.\\
Click `Set active' and close the dialog.}
\item{Next to Paparazzi you will need some additional tools. Install ffmpeg, vlc, cmake, jstest-gtk and java using:
\begin{lstlisting}[style=Bash]
sudo apt install ffmpeg vlc cmake jstest-gtk default-jre
\end{lstlisting}
}
}
\item{Install the Gazebo simulator. Gazebo version 9 is recommended for Ubuntu 18.04, earlier versions of Ubuntu may require Gazebo version 8. Install Gazebo 9 using:
\begin{lstlisting}[style=Bash]
sudo apt install gazebo9 libgazebo9-dev
\end{lstlisting}
}
\item{Install eclipse to easily navigate through the paparazzi source code.
\begin{enumerate}
\item{Download the eclipse-installer from \url{https://www.eclipse.org/downloads/download.php?file=/oomph/epp/neon/R2a/eclipse-inst-linux64.tar.gz}}
\item{Navigate to the Downloads directory, extract the .tar.gz file and run eclipse-inst.}
\item{Select the C/C++ version}
\item{It is recommended to use the default installation directory}
\item{After eclipse is installed start eclipse}
\item{Navigate to "File - New - Makefile Project with Existing Code}
\item{Name the project `paparazzi', select the paparazzi installation location (\verb|~/paparazzi|) and keep the default options}
\end{enumerate}
}
\item{Build OpenCV for the Bebop.
\begin{enumerate}
\item{Navigate to paparazzi/sw/ext/opencv\_bebop:
\begin{lstlisting}[style=Bash]
cd ~/paparazzi/sw/ext/opencv_bebop
\end{lstlisting}}
\item{Make openCV using:
\begin{lstlisting}[style=Bash]
make
\end{lstlisting}}
\end{enumerate}
}
\end{enumerate}


\section{Running your first simulation}
To test your Paparazzi installation, you will perform a short test-flight in simulation.
Navigate to the Paparazzi folder and launch the Paparazzi Center using
\begin{lstlisting}[style=Bash]
./paparazzi
\end{lstlisting}
In the top left drop-down box labeled `A/C', select the `bebop\_orange\_avoid' aircraft.
In the top middle under `Target', select `nps', the Paparazzi simulator.
Click `Clean', then `Build' to compile the example code.
Under `Session', select `Simulation - Gazebo' and click `Execute'. You should now see the Ground Control Station and the Gazebo simulator.
Click `Stop/Remove All Processes' in the Paparazzi Center to end the simulation.

\end{document}